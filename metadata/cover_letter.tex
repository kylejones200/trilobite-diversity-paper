\documentclass[11pt,a4paper]{letter}

\usepackage[utf8]{inputenc}
\usepackage[margin=1in]{geometry}
\usepackage{setspace}

\signature{Kyle T. Jones}
\address{American University \\ Washington, DC, USA \\ kylejones@american.edu}

\begin{document}

\begin{letter}{Editor \\ Journal of Asian Earth Sciences}

\opening{Dear Editor,}

We are pleased to submit our manuscript entitled \textit{``Stratigraphic Patterns of Cambrian Trilobite Diversity in Eastern Asia: A Middle Cambrian Diversity Peak''} for consideration for publication in the Journal of Asian Earth Sciences.

This manuscript addresses regional geology and paleobiology of Asia, specifically Cambrian stratigraphy and biostratigraphy in eastern Asia. The work examines trilobite diversity patterns across 98 formations and 149 biostratigraphic zones in China, Korea, and India, providing insights into middle Cambrian faunal dynamics and paleobiogeography. The study integrates lithostratigraphic and biostratigraphic data to document a diversity peak at 498.8 million years ago, corresponding to the Linguagnostus reconditus zone, and demonstrates substantial faunal differentiation between regions. This research fits the scope of JAES as it contributes to understanding the geological and paleontological history of Asia.

The main contributions of this work include documentation of a middle Cambrian diversity peak at 498.8 Ma that remains robust after rarefaction analysis, confirming that the pattern is not solely an artifact of sampling bias. Regional comparison reveals substantial faunal differentiation between China, Korea, and India, with only 19 shared taxa between China and India despite geographic proximity, indicating distinct biogeographic provinces. The study integrates formation-based and zone-based diversity metrics to provide complementary perspectives on trilobite diversity, demonstrating the utility of combining lithostratigraphic and biostratigraphic data for regional paleobiogeographic analysis. Hierarchical clustering identifies distinct middle and early Cambrian faunal groups, providing a quantitative framework for understanding faunal associations and biogeographic patterns. Finally, turnover rate analysis shows elevated origination and extinction rates around the diversity peak, consistent with broader middle Cambrian faunal changes documented in global compilations.

These findings contribute to understanding Cambrian evolution of life in Asia, regional biostratigraphic correlation, and the factors controlling trilobite diversity patterns during this critical interval in Earth history.


Thank you for considering our manuscript for publication in the Journal of Asian Earth Sciences. We look forward to your response.

\closing{Sincerely,}


\end{letter}

\end{document}

